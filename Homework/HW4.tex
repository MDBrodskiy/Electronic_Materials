%%%%%%%%%%%%%%%%%%%%%%%%%%%%%%%%%%%%%%%%%%%%%%%%%%%%%%%%%%%%%%%%%%%%%%%%%%%%%%%%%%%%%%%%%%%%%%%%%%%%%%%%%%%%%%%%%%%%%%%%%%%%%%%%%%%%%%%%%%%%%%%%%%%%%%%%%%%%%%%%%%%
% Written By Michael Brodskiy
% Class: Electronic Materials
% Professor: J. Adams
%%%%%%%%%%%%%%%%%%%%%%%%%%%%%%%%%%%%%%%%%%%%%%%%%%%%%%%%%%%%%%%%%%%%%%%%%%%%%%%%%%%%%%%%%%%%%%%%%%%%%%%%%%%%%%%%%%%%%%%%%%%%%%%%%%%%%%%%%%%%%%%%%%%%%%%%%%%%%%%%%%%

\documentclass[12pt]{article} 
\usepackage{alphalph}
\usepackage[utf8]{inputenc}
\usepackage[russian,english]{babel}
\usepackage{titling}
\usepackage{amsmath}
\usepackage{graphicx}
\usepackage{enumitem}
\usepackage{amssymb}
\usepackage[super]{nth}
\usepackage{expl3}
\usepackage[version=4]{mhchem}
\usepackage{hpstatement}
\usepackage{chemfig}
\usepackage{everysel}
\usepackage{ragged2e}
\usepackage{geometry}
\usepackage{multicol}
\usepackage{fancyhdr}
\usepackage{cancel}
\usepackage{siunitx}
\usepackage{physics}
\usepackage{tikz}
\usepackage{mathdots}
\usepackage{yhmath}
\usepackage{cancel}
\usepackage{color}
\usepackage{colortbl}
\usepackage{array}
\usepackage{multirow}
\usepackage{gensymb}
\usepackage{tabularx}
\usepackage{extarrows}
\usepackage{booktabs}
\usepackage{lastpage}
\usetikzlibrary{fadings}
\usetikzlibrary{patterns}
\usetikzlibrary{shadows.blur}
\usetikzlibrary{shapes}

\geometry{top=1.0in,bottom=1.0in,left=1.0in,right=1.0in}
\newcommand{\subtitle}[1]{%
  \posttitle{%
    \par\end{center}
    \begin{center}\large#1\end{center}
    \vskip0.5em}%

}
\usepackage{hyperref}
\hypersetup{
colorlinks=true,
linkcolor=blue,
filecolor=magenta,      
urlcolor=blue,
citecolor=blue,
}


\title{Homework 4}
\date{\today}
\author{Michael Brodskiy\\ \small Professor: J. Adams}

\begin{document}

\maketitle

\begin{enumerate}

  \item

    \begin{enumerate}

      \item We can write the maximum as:

        $$E_g=\frac{hc}{\lambda}$$

        Using our known information, we can express this as:

        $$E_g=\frac{(4.136\cdot10^{-15})(3\cdot10^{8})}{600\cdot10^{-9}}$$

        Using a solver, we obtain:

        $$\boxed{E_g=2.068[\si{eV}]=3.31\cdot10^{-19}[\si{\joule}]}$$

      \item Given the cross-sectional area ($A$) and light intensity ($I$), we know:

        $$P=IA$$

        Which then gives us:

        $$P=(5\cdot10^{-2})(2)$$
        $$\boxed{P=.1[\si{\milli\watt}]}$$

        We can then calculate the number of pairs as:

        $$N=\frac{P}{E_{ph}}$$

        This gives us:

        $$N=\frac{.1\cdot10^{-3}}{3.31\cdot10^{-19}}$$
        $$N=3.0211\cdot10^{14}\left[\si{photons\over\second}\right]$$

        Since each photon creates one pair, we get:

        $$\boxed{E=3.0211\cdot10^{14}\left[\si{electrons\over\second}\right]}$$

      \item From the same formula as part (a), we may rearrange to get:

        $$\lambda=\frac{hc}{E_g}$$

        We then plug in our known values to get:

        $$\lambda=\frac{(4.136\cdot10^{-15})(3\cdot10^{8})}{1.42}$$

        This gives us:

        $$\boxed{\lambda=873.8[\si{\nano\meter}]}$$

      \item Since light is visible from roughly 380 to 700 nanometers, this is not in the visible range (it is in the infrared range)

      \item We can find the energy gap of the given material to be $E_g=1.1[\si{eV}]$. Thus, we may calculate the wavelength as:

        $$\lambda=\frac{(4.136\cdot10^{-15})(3\cdot10^{8})}{1.1}$$

        This gives us:

        $$\lambda_{\ce{GaAs}}=1.128[\si{\micro\meter}]$$

        Since the cutoff wavelength of the given material is higher than that of the aforementioned set up, we may conclude that this detector \underline{will be able to detect} the $873.8[\si{\nano\meter}]$ wavelength.

    \end{enumerate}

  \item From the table, we may obtain the following important values:

    $$E_g=.66[\si{eV}]$$
    $$m_h^*=.40m_e$$
    $$m_e^*=.56m_e$$

    Then, we may use our formula for the density of states:

    $$N_C=2\left[\frac{2\pi m_e^*k_BT}{h^2}\right]^{\frac{3}{2}}$$
    $$N_V=2\left[\frac{2\pi m_h^*k_BT}{h^2}\right]^{\frac{3}{2}}$$

    We may thus write:

    $$N_C=2\left[\frac{2\pi \cdot.56\cdot9.1\cdot10^{-31}\cdot 1.38\cdot10^{-23}\cdot300}{(6.626\cdot10^{-34})^2}\right]^{\frac{3}{2}}$$
    $$N_V=2\left[\frac{2\pi \cdot.4\cdot9.1\cdot10^{-31}\cdot 1.38\cdot10^{-23}\cdot300}{(6.626\cdot10^{-34})^2}\right]^{\frac{3}{2}}$$

    We calculate to get:

    $$\boxed{N_C=1.0493\cdot10^{25}\left[ \si{\per\meter\cubed} \right]}$$
    $$\boxed{N_V=6.3343\cdot10^{24}\left[ \si{\per\meter\cubed} \right]}$$

    We then use the expression for the intrinsic carrier concentration:

    $$n_i=\sqrt{N_CN_V}e^{-\frac{E_g}{2k_BT}}$$

    Using our values:

    $$n_i=\sqrt{(1.0493\cdot10^{25})(6.3343\cdot10^{24})}e^{-\frac{.66}{2(8.6\cdot10^{-5})(300)}}$$

    This gives us:

    $$\boxed{n_i=2.2718\cdot10^{19}[\si{\per\meter\cubed}]}$$

    Using the values from the table, we may get:

    $$n_i=\sqrt{(1.04\cdot10^{25})(6\cdot10^{24})}e^{-\frac{.66}{2(8.6\cdot10^{-5})(300)}}$$
    $$\boxed{n_i=2.2012\cdot10^{19}[\si{\per\meter\cubed}]}$$

    Finally, we can use this to calculate the intrinsic resistivity. We know that:

    $$\rho=\frac{1}{\sigma}$$

    And that:

    $$\sigma=en_1(\mu_c+\mu_h)$$

    We then get:

    $$\rho=\frac{1}{(1.6\cdot10^{-19})(2.2012\cdot10^{19})(.39+.19)}$$
    $$\rho=(2.0427)^{-1}$$
    $$\boxed{\rho=.4895[\si{\ohm\meter}]}$$

    Note that, using our calculated intrinsic carrier concentration we obtain a slightly different value:

    $$\rho=\frac{1}{(1.6\cdot10^{-19})(2.2718\cdot10^{19})(.39+.19)}$$
    $$\rho=(2.1082)^{-1}$$
    $$\boxed{\rho=.4743[\si{\ohm\meter}]}$$

  \item We can express the formula for Fermi level as:

    $$\Delta E=\frac{k_BT}{2}\ln\left( \frac{m_h^*}{m_e^*} \right)$$

    We expand $\Delta E$ to:

    $$\Delta E=E_f-E_i$$

    With $E_f$ representing the Fermi level and $E_i$ representing the intrinsic (middle of the bandgap) Fermi level. As such, we write:

    $$E_f-E_i=\frac{k_BT}{2}\ln\left( \frac{m_h^*}{m_e^*} \right)$$

    We can write $m_h^*$ and $m_e^*$ from the table to write the ratios for each material:

    $$\ce{Ge}\to\frac{.4}{.56}=.7143$$
    $$\ce{Si}\to\frac{.6}{1.08}=.5556$$
    $$\ce{GaAs}\to\frac{.5}{.067}=7.4627$$

    We then apply this to the formula:

    $$\ce{Ge}\to\frac{(8.617\cdot10^{-5}\cdot300)}{2}\ln(.7143)$$
    $$\ce{Si}\to\frac{(8.617\cdot10^{-5}\cdot300)}{2}\ln(.5556)$$
    $$\ce{GaAs}\to\frac{(8.617\cdot10^{-5}\cdot300)}{2}\ln(7.4627)$$

    This gives us:

    $$\Delta E_{\ce{Ge}}=-4.3488\cdot10^{-3}[\si{eV}]$$
    $$\Delta E_{\ce{Si}}=-7.5964\cdot10^{-3}[\si{eV}]$$
    $$\Delta E_{\ce{GaAs}}=.025979[\si{eV}]$$

    Given that we may rewrite with the intrinsic energy band gap, we may observe that Germanium and Silicon are below the intrinsic band gap by -4.3488 and -7.5964 milli-electron-volts, respectively, and Gallium-Arsenide .025979 electron-volts above the intrinsic band gap. We can calculate the precise Fermi level as:

    $$E_{f,\ce{Ge}}=.33-4.3488\cdot10^{-3}[\si{eV}]$$
    $$E_{f,\ce{Si}}=.55-7.5964\cdot10^{-3}[\si{eV}]$$
    $$E_{f,\ce{GaAs}}=.71+.025979[\si{eV}]$$

    Finally, we find:

    $$\boxed{E_{f,\ce{Ge}}=.3257[\si{eV}]}$$
    $$\boxed{E_{f,\ce{Si}}=.5424[\si{eV}]}$$
    $$\boxed{E_{f,\ce{GaAs}}=.736[\si{eV}]}$$

  \item First and foremost, we are given:

    $$N_D=10^{15}[\si{\per\centi\meter\cubed}]$$

    From the tables from the problems above, we may obtain:

    $$\mu_e=1350\left[\si{\centi\meter\squared\over\volt\second}\right]$$

    We can apply our formula for conductivity to get:

    $$\sigma=eN_D\mu_e$$

    This gives us:

    $$\sigma=(1.6\cdot10^{-19})(1350)(10^{15})$$
    $$\boxed{\sigma=.216[\si{\per\ohm\centi\meter}]=21.6[\si{\per\ohm\meter}]}$$

    We then find:

    $$\rho=\frac{1}{\sigma}$$
    $$\rho=\frac{1}{21.6}$$
    $$\boxed{\rho=.046296[\si{\ohm\meter}]}$$

\end{enumerate}

\end{document}

