%%%%%%%%%%%%%%%%%%%%%%%%%%%%%%%%%%%%%%%%%%%%%%%%%%%%%%%%%%%%%%%%%%%%%%%%%%%%%%%%%%%%%%%%%%%%%%%%%%%%%%%%%%%%%%%%%%%%%%%%%%%%%%%%%%%%%%%%%%%%%%%%%%%%%%%%%%%%%%%%%%%
% Written By Michael Brodskiy
% Class: Electronic Materials
% Professor: J. Adams
%%%%%%%%%%%%%%%%%%%%%%%%%%%%%%%%%%%%%%%%%%%%%%%%%%%%%%%%%%%%%%%%%%%%%%%%%%%%%%%%%%%%%%%%%%%%%%%%%%%%%%%%%%%%%%%%%%%%%%%%%%%%%%%%%%%%%%%%%%%%%%%%%%%%%%%%%%%%%%%%%%%

\include{Includes.tex}

\title{Homework 5}
\date{\today}
\author{Michael Brodskiy\\ \small Professor: J. Adams}

\begin{document}

\maketitle

\begin{enumerate}

    \setcounter{enumi}{8}

  \item

    \begin{enumerate}[label=\alph*.]

      \item 

        \begin{enumerate}[label=\arabic*.]

          \item To find the conductivity, we may use the formula:

            $$\sigma=ne\mu_e$$

            Where $n$ represents the electron concentration, $e$ represents the charge of an electron, and $\mu_e$ is the electron mobility. Given that the mobility of electrons in silicon is:

            $$\mu_{e,27[\si{\celsius}]}=8\cdot10^2\left[ \si{\centi\meter\squared\over\volt\second} \right]$$

            We may substitute to get:

            $$\sigma=\left( 1\cdot10^{17} \right)\left( 1.6\cdot10^{-19} \right)(800)$$

            This gives us:

            $$\boxed{\sigma_{27[\si{\celsius}]}=12.8\left[ \si{\siemens\over\centi\meter} \right]}$$

          \item In intrinsic silicon, we know that:

            $$n_i=N_ce^{-\frac{E_c-E_{fi}}{k_BT}}$$

            In $n$-type doped silicon, we take $n\to N_d$ to get:

            $$N_d=N_ce^{-\frac{E_c-E_{fn}}{k_BT}}$$

            We divide the doped equation by the original to find a ratio:

            $$\frac{N_d}{n_i}=e^{\frac{E_{fn}-E_{fi}}{k_BT}}$$

            We can simplify to get:

            $$\ln\left(\frac{N_d}{n_i}\right)=\frac{E_{fn}-E_{fi}}{k_BT}$$

            We can take the difference in Fermi energies as:

            $$\Delta E_f=E_{fn}-E_{fi}$$

            Using this, we get:

            $$\Delta E_f=k_BT\ln\left(\frac{N_d}{n_i}\right)$$

            We can then substitute our known values to find:

            $$\Delta E_f=\left( 8.617\cdot10^{-5} \right)(300)\ln\left(\frac{10^{17}}{1.45\cdot10^{10}}\right)$$
            $$\boxed{\Delta E_f=.407[\si{eV}]}$$

            We see that the Fermi energy of the $n$-type doped silicon is .407$[\si{eV}]$ higher than intrinsic silicon

          \item At $T=127[\si{\celsius}]=400[\si{\kelvin}]$, we may find the mobility to be:

            $$\mu_{e,127[\si{\celsius}]}=4.5\cdot10^2\left[ \si{\centi\meter\squared\over\volt\second} \right]$$

            This gives us:

            $$\sigma=\left( 10^{17} \right)(1.6\cdot10^{-19})(450)$$

            And finally:

            $$\boxed{\sigma_{127[\si{\celsius}]}=7.2\left[ \si{\siemens\over\centi\meter} \right]}$$

        \end{enumerate}

      \item 

        \begin{enumerate}[label=\arabic*.]

          \item Similar to part (a), we may write:

            $$\sigma=ne\mu_e$$

            We note the net effect of doping as:

            $$n=N_d-N-a$$
            $$n=10^{17}-9\cdot10^{16}$$
            $$n=1\cdot10^{16}\left[ \si{\per\centi\meter\cubed} \right]$$

            Thus, we see that this is still an $n$-type semiconductor. We may note, however, that electron scattering occurs, and, thus, the drift mobility becomes:

            $$\mu_{e,27[\si{\celsius}]}=7\cdot10^2\left[ \si{\centi\meter\squared\over\volt\second} \right]$$

            We then plug in our values to get:

            $$\sigma=(10^{16})(1.6\cdot10^{-19})(700)$$
            
            Now we get:

            $$\boxed{\sigma_{n,p,27[\si{\celsius}]}=1.12\left[ \si{\siemens\over\centi\meter} \right]}$$

          \item Using the same formula from part (a), except now with $n=N_d-N_a$, we may get:

            $$\Delta E_f=k_BT\ln\left( \frac{N_d-N_a}{n_i} \right)$$

            This gives us:

            $$\Delta E_f=\left( 8.617\cdot10^{-5} \right)(300)\ln\left( \frac{10^{16}}{1.45\cdot10^{10}} \right)$$
            $$\boxed{\Delta E_f=3.475[\si{eV}]}$$

            We may notice that the Fermi energy difference has shrunk.

        \end{enumerate}

    \end{enumerate}

    \setcounter{enumi}{5}

  \item

    \begin{enumerate}[label=\alph*]

      \item \underline{No, doping will not always increase the conductivity}. For example, take the intrinsic state with $n=p$. Let us then take a $p$-type material (say, Boron), and use it to slightly dope. This will result in $p>n$, which would mean more holes with lower mobility (as opposed to electrons with higher mobility), and, as such, the conductivity would decrease.

      \item We may begin by applying the mass-action law:

        $$np=n_i^2$$

        We isolate $n$ to write:

        $$n=\frac{n_i^2}{p}$$

        We substitute this into the given conductivity equation to get:

        $$\sigma=\frac{en_i^2\mu_e}{p}+pe\mu_h$$

        We want to find the minimum $p$ value, so we proceed to differentiate:

        $$\frac{\partial\sigma}{\partial p}=-\frac{en_i^2\mu_e}{p^2}+e\mu_h$$

        We want to find points at which the sign changes, so we set the partial equal to 0 to get:

        $$-\frac{en_i^2\mu_e}{p^2}+e\mu_h=0$$

        We can then solve for $p$:

        $$\frac{en_i^2\mu_e}{p^2}=e\mu_h$$
        $$\frac{en_i^2\mu_e}{e\mu_h}=p^2$$
        $$p=\pm\sqrt{\frac{en_i^2\mu_e}{e\mu_h}}$$

        Of course, $p$ can not physically be negative, so we get:

        $$\boxed{p=n_i\sqrt{\frac{\mu_e}{\mu_h}}}$$

        From here, we can substitute this into our conductivity equation to get:

        $$\sigma=\frac{en_i\mu_e}{\sqrt{\frac{\mu_e}{\mu_h}}}+n_i\sqrt{\frac{\mu_e}{\mu_h}}e\mu_h$$

        This gives us a simplified form as:

        $$\sigma_{min}=e\left[n_i\sqrt{\mu_e\mu_h}+n_i\sqrt{\mu_e\mu_h}\right]$$
        $$\boxed{\sigma_{min}=2en_i\sqrt{\mu_e\mu_h}}$$

      \item From Table 5.1, we may get:

        $$\mu_{e}=1.35\cdot10^3\left[ \si{\centi\meter\squared\over\volt\second} \right]$$
        $$\mu_{h}=4.5\cdot10^2\left[ \si{\centi\meter\squared\over\volt\second} \right]$$

        Using the equations obtained in (b) we get:

        $$p_m=(1.45\cdot10^{10})\sqrt{\frac{1.35}{.45}}$$
        $$\boxed{p_m=2.512\cdot10^{10}\left[ \si{\per\centi\meter\cubed} \right]}$$

        We then take our conductivity equation as:

        $$\sigma_{min}=2(1.6\cdot10^{-19})(1.45\cdot10^{10})\sqrt{(1.35\cdot10^3)(.45\cdot10^3)}$$
        $$\boxed{\sigma_{min}=3.617\left[ \si{\siemens\over\centi\meter} \right]}$$

        We know the intrinsic value of $p$, so we take the ratio to get:

        $$\frac{p_m}{n_i}=\frac{2.512}{1.45}$$
        $$\frac{p_m}{n_i}=1.7324$$

        We can then find the intrinsic conductivity as:

        $$\sigma_i=\left( 1.6\cdot10^{-19} \right)\left( 1.45\cdot10^{10} \right)\left[ 1.35+.45 \right]\cdot10^{3}$$
        $$\sigma_i=4.176\cdot10^{-6}[\si{\siemens\over\centi\meter}]$$

        We then take the ratio to write:

        $$\frac{\sigma_{min}}{\sigma_i}=\frac{3.617}{4.176}$$
        $$\frac{\sigma_{min}}{\sigma_i}=.8661$$

        We may see that $p$-type doping to the quantity calculated above results in a 73.24\% increase in hole concentration, with a 13.39\% decrease in conductivity.

    \end{enumerate}

  \item We may begin by writing:

    $$\sigma=pq\mu_h$$

    We then convert this to resistivity:

    $$\rho=[pq\mu_h]^{-1}$$

    We can substitute known values (including $p\to N_{dopant}$) to get:

    $$\rho=(1.6\cdot10^{-19})^{-1}N_{dopant}^{-1}\left[ \frac{1+3.745\cdot10^{-18}N_{dopant}}{461.3+2.0335\cdot10^{-18}N_{dopant}} \right]$$

    We take $\rho\to.1$ and then solve:

    $$.1=(1.6\cdot10^{-19})^{-1}N_{dopant}^{-1}\left[ \frac{1+3.745\cdot10^{-18}N_{dopant}}{461.3+2.0335\cdot10^{-16}N_{dopant}} \right]$$
    $$1.6\cdot10^{-20}N_{dopant}=\left[ \frac{1+3.745\cdot10^{-18}N_{dopant}}{461.3+2.0335\cdot10^{-16}N_{dopant}} \right]$$
    $$7.3808\cdot10^{-18}N_{dopant}+3.2536\cdot10^{-36}N_{dopant}^2=1+3.745\cdot10^{-18}N_{dopant}$$
    $$3.2536\cdot10^{-36}N_{dopant}^2+3.6358\cdot10^{-18}N_{dopant}-1=0$$

    We can put this in quadratic form:

    $$N_{dopant}^2+1.175\cdot10^{18}N_{dopant}-3.0735\cdot10^{35}=0$$

    Solving this gives us:

    $$N_{dopant}=2.2028535\cdot10^{17},-1.3952853\cdot10^{18}$$

    Since we know the quantity must be positive, we conclude:

    $$\boxed{N_{dopant}=2.2028535\cdot10^{17}[\si{\per\centi\meter\cubed}]}$$

    \setcounter{enumi}{11}

  \item

    \begin{enumerate}[label=\alph*]

      \item Per the problem, we are given:

        $$N_D=10^{17}[\si{\per\centi\meter\cubed}]$$

        We use this in the given formulas to get:

        $$\mu_e=88+\frac{1252}{1+6.984\cdot10^{-18}(10^{17})}$$

        This gives us:

        $$\mu_e=825.16\left[\si{\centi\meter\squared\over\volt\second}\right]$$

        We then calculate:

        $$\sigma=ne\mu_e$$
        $$\sigma=(10^{17})(1.6\cdot10^{-19})(825.16)$$
        $$\boxed{\sigma=13.203\left[ \si{\siemens\over\centi\meter} \right]}$$

      \item Similar to part (1), we write:

        $$N_D=10^{17}[\si{\per\centi\meter\cubed}]$$
        $$\mu_h=54.3+\frac{407}{1+3.745\cdot10^{-18}(10^{17})}$$

        This gives us:

        $$\mu_h=350.41\left[\si{\centi\meter\squared\over\volt\second}\right]$$

        From here, we may write:

        $$\sigma=pq\mu_h$$

        We take $p\to N_A$ to get:

        $$13.203=N_A(1.6\cdot10^{-19})(350.41)$$

        This gives us:

        $$N_A=\frac{13.203}{350.41(1.6\cdot10^{-19})}$$
        $$\boxed{N_A=2.3548\cdot10^{17}[\si{\per\centi\meter\cubed}]}$$

    \end{enumerate}

\end{enumerate}

\end{document}

