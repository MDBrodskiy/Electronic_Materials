%%%%%%%%%%%%%%%%%%%%%%%%%%%%%%%%%%%%%%%%%%%%%%%%%%%%%%%%%%%%%%%%%%%%%%%%%%%%%%%%%%%%%%%%%%%%%%%%%%%%%%%%%%%%%%%%%%%%%%%%%%%%%%%%%%%%%%%%%%%%%%%%%%%%%%%%%%%%%%%%%%%
% Written By Michael Brodskiy
% Class: Electronic Materials
% Professor: J. Adams
%%%%%%%%%%%%%%%%%%%%%%%%%%%%%%%%%%%%%%%%%%%%%%%%%%%%%%%%%%%%%%%%%%%%%%%%%%%%%%%%%%%%%%%%%%%%%%%%%%%%%%%%%%%%%%%%%%%%%%%%%%%%%%%%%%%%%%%%%%%%%%%%%%%%%%%%%%%%%%%%%%%

\documentclass[12pt]{article} 
\usepackage{alphalph}
\usepackage[utf8]{inputenc}
\usepackage[russian,english]{babel}
\usepackage{titling}
\usepackage{amsmath}
\usepackage{graphicx}
\usepackage{enumitem}
\usepackage{amssymb}
\usepackage[super]{nth}
\usepackage{expl3}
\usepackage[version=4]{mhchem}
\usepackage{hpstatement}
\usepackage{chemfig}
\usepackage{everysel}
\usepackage{ragged2e}
\usepackage{geometry}
\usepackage{multicol}
\usepackage{fancyhdr}
\usepackage{cancel}
\usepackage{siunitx}
\usepackage{physics}
\usepackage{tikz}
\usepackage{mathdots}
\usepackage{yhmath}
\usepackage{cancel}
\usepackage{color}
\usepackage{colortbl}
\usepackage{array}
\usepackage{multirow}
\usepackage{gensymb}
\usepackage{tabularx}
\usepackage{extarrows}
\usepackage{booktabs}
\usepackage{lastpage}
\usetikzlibrary{fadings}
\usetikzlibrary{patterns}
\usetikzlibrary{shadows.blur}
\usetikzlibrary{shapes}

\geometry{top=1.0in,bottom=1.0in,left=1.0in,right=1.0in}
\newcommand{\subtitle}[1]{%
  \posttitle{%
    \par\end{center}
    \begin{center}\large#1\end{center}
    \vskip0.5em}%

}
\usepackage{hyperref}
\hypersetup{
colorlinks=true,
linkcolor=blue,
filecolor=magenta,      
urlcolor=blue,
citecolor=blue,
}


\title{Homework 2}
\date{\today}
\author{Michael Brodskiy\\ \small Professor: J. Adams}

\begin{document}

\maketitle

\begin{enumerate}

  \item

    \begin{enumerate}

      \item We begin by calculating the inter-planar spacing as:

        $$d_{hkl}=\frac{a}{\sqrt{h^2+k^2+l^2}}$$
        $$d_{110}=\frac{.2866}{\sqrt{1+1+0}}$$
        $$\boxed{d_{110}=.2027[\si{\nano\meter}]}$$

        We may write Bragg's law as:

        $$n\lambda=2d\sin(\theta)$$

        We take $n\to1$ since we are interested in the first diffraction angle. This gives us:

        $$(.154)=2(.2027)\sin(\theta)$$
        $$2\theta_{110}=2\sin^{-1}\left( \frac{.154}{2(.2027)} \right)$$
        $$\boxed{2\theta_{110}=44.65^{\circ}}$$

        We repeat this process for the other two planes:
        
        $$d_{200}=\frac{.2866}{\sqrt{4+0+0}}$$
        $$\boxed{d_{200}=.1433}$$

        $$d_{211}=\frac{.2866}{\sqrt{4+1+1}}$$
        $$\boxed{d_{211}=.117}$$

        We then calculate the angles:

        $$2\theta_{200}=2\sin^{-1}\left( \frac{.154}{2(.1433)} \right)$$
        $$\boxed{2\theta_{200}=65^{\circ}}$$

        $$2\theta_{211}=2\sin^{-1}\left( \frac{.154}{2(.117)} \right)$$
        $$\boxed{2\theta_{211}=82.31^{\circ}}$$

      \item 

      \item 

    \end{enumerate}

  \item

    \begin{enumerate}

      \item Given such intersections, we may find the Miller indices to be:

        $$h=a\left( \frac{1}{a} \right)\quad k=a\left( \frac{2}{a} \right)\quad l=a(0)$$

        This gives us Miller Indices of:

        $$\boxed{(120)}$$

      \item Symmetry-equivalent planes have Miller Indices whose values create the same inter-planar spacing. This can be found by finding the scale factor for the Miller Indices from part (a):

        $$\sqrt{1^2+2^2+0}=\sqrt{5}$$

        We then find symmetry-equivalent planes to be those which have this same scale factor. Thus, we may find that the following are symmetry-equivalent:

        $$\boxed{(210), (201), (102), (021), (012)}$$

      \item 

    \end{enumerate}

  \item

    \begin{enumerate}

      \item 

      \item 

      \item 

    \end{enumerate}

  \item

    \begin{enumerate}

      \item 

      \item 

      \item 

    \end{enumerate}

\end{enumerate}

\end{document}

