%%%%%%%%%%%%%%%%%%%%%%%%%%%%%%%%%%%%%%%%%%%%%%%%%%%%%%%%%%%%%%%%%%%%%%%%%%%%%%%%%%%%%%%%%%%%%%%%%%%%%%%%%%%%%%%%%%%%%%%%%%%%%%%%%%%%%%%%%%%%%%%%%%%%%%%%%%%%%%%%%%%
% Written By Michael Brodskiy
% Class: Electronic Materials
% Professor: J. Adams
%%%%%%%%%%%%%%%%%%%%%%%%%%%%%%%%%%%%%%%%%%%%%%%%%%%%%%%%%%%%%%%%%%%%%%%%%%%%%%%%%%%%%%%%%%%%%%%%%%%%%%%%%%%%%%%%%%%%%%%%%%%%%%%%%%%%%%%%%%%%%%%%%%%%%%%%%%%%%%%%%%%

\documentclass[12pt]{article} 
\usepackage{alphalph}
\usepackage[utf8]{inputenc}
\usepackage[russian,english]{babel}
\usepackage{titling}
\usepackage{amsmath}
\usepackage{graphicx}
\usepackage{enumitem}
\usepackage{amssymb}
\usepackage[super]{nth}
\usepackage{expl3}
\usepackage[version=4]{mhchem}
\usepackage{hpstatement}
\usepackage{chemfig}
\usepackage{everysel}
\usepackage{ragged2e}
\usepackage{geometry}
\usepackage{multicol}
\usepackage{fancyhdr}
\usepackage{cancel}
\usepackage{siunitx}
\usepackage{physics}
\usepackage{tikz}
\usepackage{mathdots}
\usepackage{yhmath}
\usepackage{cancel}
\usepackage{color}
\usepackage{colortbl}
\usepackage{array}
\usepackage{multirow}
\usepackage{gensymb}
\usepackage{tabularx}
\usepackage{extarrows}
\usepackage{booktabs}
\usepackage{lastpage}
\usetikzlibrary{fadings}
\usetikzlibrary{patterns}
\usetikzlibrary{shadows.blur}
\usetikzlibrary{shapes}

\geometry{top=1.0in,bottom=1.0in,left=1.0in,right=1.0in}
\newcommand{\subtitle}[1]{%
  \posttitle{%
    \par\end{center}
    \begin{center}\large#1\end{center}
    \vskip0.5em}%

}
\usepackage{hyperref}
\hypersetup{
colorlinks=true,
linkcolor=blue,
filecolor=magenta,      
urlcolor=blue,
citecolor=blue,
}


\title{Lecture 2 — Atomic Structure}
\date{\today}
\author{Michael Brodskiy\\ \small Professor: J. Adams}

\begin{document}

\maketitle

\begin{itemize}

  \item Dalton's Atomic Theory (1803)

    \begin{itemize}

      \item Key Postulates:

        \begin{itemize}

          \item All matter is made up of tiny, indivisible particles called atoms

          \item Atoms of a given element are identical in size, mass, and properties

          \item Atoms of different elements differ in these properties

        \end{itemize}

    \end{itemize}

  \item Thomson's Plum Pudding Model

    \begin{itemize}

      \item Cathode Ray Experiment (1897)

        \begin{itemize}

          \item Observed that cathode rays are streams of negatively charged particles

          \item Discovery of the electron, the first subatomic particle

        \end{itemize}

      \item Plum Pudding Model

        \begin{itemize}

          \item Proposed by J.J. Thomson

          \item Atoms consist of a positively charged ``pudding'' with negatively charged electrons embedded within, like plums in a pudding

        \end{itemize}

      \item Significance:

        \begin{itemize}

          \item Challenged Dalton's idea of indivisible atoms

          \item Demonstrated that atoms have internal structure
            
        \end{itemize}

      \item Limitations:

        \begin{itemize}

          \item Could not explain the distribution of charge or atomic structure

        \end{itemize}

    \end{itemize}

  \item Rutherford's Gold Foil Experiment (1911)

  \item Bohr's Model (1913)

    \begin{itemize}

      \item Key Features:

        \begin{itemize}

          \item Electrons orbit the nucleus in fixed, quantized energy levels

          \item Electrons can move between energy levels by absorbing or emitting energy (photons)

          \item Orbits correspond to specific allowed energy states, preventing electron collapse into the nucleus

        \end{itemize}

      \item Supporting Evidence:

        \begin{itemize}

          \item Successfully explained the hydrogen emission spectrum

          \item Discrete spectral lines correspond to energy transition between levels

        \end{itemize}

    \end{itemize}

  \item Modern Quantum Mechanical Model (Wave-Particle Duality, de Broglie, Schr\"odinger)

    \begin{itemize}

      \item Key Concepts:

        \begin{itemize}

          \item Electrons exhibit wave-particle duality (de Broglie hypothesis)

          \item Electrons exist in orbitals, regions of space with a high probability of finding an electron

          \item Atomic behavior described using Schr\"odinger's equation, which defines the wave function ($\psi$)

        \end{itemize}

      \item Quantum Numbers

        \begin{itemize}

          \item Describe the unique quantum state of an electron in an atom

          \item Define energy, shape, orientation, and spin of electron orbitals

          \item Four Numbers:

            \begin{enumerate}

              \item The Principal Quantum Number ($n$)

              \item Angular Momentum Quantum Number ($l$)

              \item Magnetic Quantum Number ($m_l$)

              \item Spin Quantum Number ($m_s$)

            \end{enumerate}

        \end{itemize}
        
    \end{itemize}

  \item Atomic Principles

    \begin{itemize}

      \item Aufbau Principle:

        \begin{itemize}

          \item Electrons fill orbitals starting with the lowest energy level first

          \item Order of orbital filling: $1s\to2s\to2p\to3s\to3p\to4s\to3d\to4p$, etc.

          \item Visualize the filling sequence with the diagonal rule or energy diagram

        \end{itemize}

      \item Pauli Exclusion Principle:

        \begin{itemize}

          \item No two electrons in an atom can have the same set of all four quantum numbers ($n$, $l$, $m_l$, $m_s$)

          \item Each orbital can hold a maximum of two electrons with opposite spins

        \end{itemize}

      \item Hund's Rule:

        \begin{itemize}

          \item When electrons fill degenerate orbitals (orbitals with the same energy, \textit{e}.\textit{g}. $p$,$d$,$f$) they maximize unpaired spins before pairing

          \item Ensures the lowest-energy arrangement by minimizing electron repulsion

        \end{itemize}

      \item Significance:

        \begin{itemize}

          \item Explains electron configurations of elements

          \item Influences magnetic and electrical properties (like ferromagnetism)

        \end{itemize}

    \end{itemize}

  \item Impacts of Atomic Structure

    \begin{itemize}

      \item Bonding: Determines whether a material is metallic, covalent, or ionic

      \item Electron Configuration: Influences conductivity, magnetism, and optical properties

      \item Semiconductors: Silicon (Si) covalent bonding and band gap make it ideal for transistors

      \item Insulators: Aluminum Oxide (Al$_2$O$_3$) strong ionic bonds and high band gap prevent conductivity

    \end{itemize}

  \item Periodic Table Organization

    \begin{itemize}

      \item Structure:

        \begin{itemize}

          \item Rows (Periods):

            \begin{itemize}

              \item Indicate the principal quantum number ($n$) of the outermost electron shell

            \end{itemize}

          \item Columns (Groups):

            \begin{itemize}

              \item Elements in the same group have similar valence electron configurations, leading to similar chemical properties

            \end{itemize}

        \end{itemize}

    \end{itemize}

\end{itemize}

\end{document}

