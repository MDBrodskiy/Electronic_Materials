%%%%%%%%%%%%%%%%%%%%%%%%%%%%%%%%%%%%%%%%%%%%%%%%%%%%%%%%%%%%%%%%%%%%%%%%%%%%%%%%%%%%%%%%%%%%%%%%%%%%%%%%%%%%%%%%%%%%%%%%%%%%%%%%%%%%%%%%%%%%%%%%%%%%%%%%%%%%%%%%%%%
% Written By Michael Brodskiy
% Class: Electronic Materials
% Professor: J. Adams
%%%%%%%%%%%%%%%%%%%%%%%%%%%%%%%%%%%%%%%%%%%%%%%%%%%%%%%%%%%%%%%%%%%%%%%%%%%%%%%%%%%%%%%%%%%%%%%%%%%%%%%%%%%%%%%%%%%%%%%%%%%%%%%%%%%%%%%%%%%%%%%%%%%%%%%%%%%%%%%%%%%

\include{Includes.tex}

\title{Lecture 3 — Ionic Bonding}
\date{\today}
\author{Michael Brodskiy\\ \small Professor: J. Adams}

\begin{document}

\maketitle

\begin{itemize}

  \item What is Ionic Bonding?

    \begin{itemize}

      \item Ionic Bonding occurs when electrons are transferred from one atom to another, creating oppositely charged ions that attract each other

      \item Typically occurs between a metal (donates electrons) and a nonmetal (accepts electrons)

      \item How it happens:

        \begin{itemize}

          \item Metal (like Sodium) loses an electron $\to$ forms a positively charged ion (Na$^+$)

          \item Nonmetal (like Chlorine) gains an electron$\to$ forms a negatively charged ion (CL$^-$)

          \item Attraction: electrostatic forces hold the ions together

        \end{itemize}

    \end{itemize}

  \item Role of Electronegativity

    \begin{itemize}

      \item A measure of an atom's ability to attract and hold electrons in a bond

      \item Higher electronegativity $\to$ stronger pull on electrons

      \item Electronegativity Difference

        \begin{itemize}

          \item Ionic bonding occurs when there is a large difference in electronegativity ($>1.7$) between two atoms

          \item Metal: Low electronegativity $\to$ tends to lose electrons (for example, sodium has an electronegativity of .9)

          \item Nonmetal: High electronegativity $\to$ tends to gain electrons (for example, chlorine has an electronegativity of 3.0)

        \end{itemize}

    \end{itemize}

  \item Coulomb's Law

    \begin{itemize}

      \item Formula:

        $$\vec{F}=\frac{k|q_1q_2|}{r^2}$$

        \begin{itemize}

          \item $k$ is the Coulomb constant ($8.987\cdot10^9[\si{\newton\meter\per\coulomb\squared}]$)

          \item $q_1$ and $q_2$ are the charges on the ions

          \item $r$ is the ionic radius

          \item $F$ is the force between the two ions

        \end{itemize}

      \item Related to the lattice energy:

        $$E_L\propto \frac{k|q_1q_2|}{r}$$

      \item Explains the high melting points and stability of ionic solids

      \item Predicts relative strength of ionic compounds for different applications

    \end{itemize}

  \item Polyatomic Ions

    \begin{itemize}

      \item Groups of atoms covalently bonded together that act as a single charged unit; Examples:

        \begin{itemize}

          \item Ammonium (NH$_4^+$): Acts as a cation

          \item Nitrate (NO$_3^-$): Acts as an anion

          \item Sulfate (SO$_4^{2-}$): Double negative charge

        \end{itemize}

    \end{itemize}

  \item High Melting and Boiling Points

    \begin{itemize}
        
      \item Strong Electrostatic Force

        \begin{itemize}

          \item Ionic compounds are held together by strong electrostatic attractions between oppositely charged ions

          \item Overcoming these forces requires significant energy, resulting in high melting and boiling points

        \end{itemize}

      \item Lattice Structure and Stability

        \begin{itemize}

          \item 3D Crystal Lattice: tightly packed lattice structure maximizes attractions and minimizes repulsions, enhancing stability

          \item The lattice's stability translates into higher energy requirements for breaking bonds
        
        \end{itemize}

      \item Impact of Ionic Charge and Size

        \begin{itemize}

          \item Higher charges: Compounds with higher ionic charges have stronger attractions and higher melting/boiling points compared to singly charged ions

          \item Smaller ions: Smaller ionic radii allow ions to pack more closely, increasing bond strength and thermal stability

        \end{itemize}

    \end{itemize}

  \item Brittleness

    \begin{itemize}

      \item Nature of Ionic Bonds

        \begin{itemize}

          \item Ions in the crystal lattice are held in fixed positions by strong electrostatic forces

        \end{itemize}

      \item Response to Stress

        \begin{itemize}

          \item When force is applied, layers of ions shift relative to each other

          \item This causes like-charged ions to align, resulting in strong repulsive forces

          \item The repulsion causes the lattice to fracture

        \end{itemize}

    \end{itemize}

  \item Electrical Conductivity

    \begin{itemize}

      \item Conductivity in Different States

        \begin{itemize}

          \item Solid State

            \begin{itemize}

              \item Ionic compounds do not conduct electricity in the solid state

              \item Reason: Ions are fixed in the crystal lattice and can not move freely

            \end{itemize}

          \item Molten or Dissolved State

            \begin{itemize}

              \item Ionic compounds conduct electricity when melted or dissolved in water

              \item Reason: Ions are free to move, allowing the flow of charge

            \end{itemize}

        \end{itemize}

    \end{itemize}

  \item Solubility in Polar Solvents

    \begin{itemize}

      \item Ionic Compounds are Soluble in Polar Solvents

        \begin{itemize}

          \item Polar Nature of the Solvent

            \begin{itemize}

              \item Polar solvents like water have molecules with partial positive ($\delta+$) and partial negative ($\delta-$) charges

              \item These charges interact with the ions in the ionic compound, breaking the lattice apart

            \end{itemize}

          \item Ion-Dipole Interaction

            \begin{itemize}

              \item Positive ions are surrounded by the partial negative charges of water molecules (oxygen)

              \item Negative ions are surrounded by the partial positive charges of water molecules (hydrogen)

            \end{itemize}

          \item Why Polar Solvents are Effective

            \begin{itemize}

              \item The strong dipole moment of water (or other polar solvents) provides the necessary energy to disrupt the ionic lattice

            \end{itemize}

        \end{itemize}

    \end{itemize}

  \item Hardness and Density

    \begin{itemize}

      \item Hardness of Ionic Compounds

        \begin{itemize}

          \item Ionic compounds are hard because of the strong electrostatic forces holding the ions in a rigid, 3D lattice structure

          \item Displacing ions requires significant energy to overcome these forces

        \end{itemize}

      \item Density of Ionic Compounds

        \begin{itemize}

          \item Ionic lattices are closely packed due to the strong attraction between ions

          \item Smaller ions or higher charges increasing packing efficiency

        \end{itemize}

    \end{itemize}

  \item Thermal Stability

    \begin{itemize}
        
      \item The ability of a compound to withstand high temperatures without decomposing or breaking down into its elements

      \item Stability of Ionic Compounds

        \begin{itemize}

          \item Strong Ionic Bonds

            \begin{itemize}

              \item The electrostatic forces between oppositely charged ions require significant energy to overcome

            \end{itemize}

          \item Lattice Energy

            \begin{itemize}
                
              \item High lattice energy contributes to stability by tightly binding ions in the lattice

            \end{itemize}

        \end{itemize}

      \item Factors Influencing Thermal Stability

        \begin{itemize}

          \item Ionic Charge

            \begin{itemize}

              \item Higher charges lead to stronger bonds and greater stability

            \end{itemize}

          \item Ionic Radius

            \begin{itemize}

              \item Smaller ions lead to closer packing and stronger bonds, enhancing stability

            \end{itemize}

        \end{itemize}

    \end{itemize}

  \item Optical Transparency

    \begin{itemize}

      \item The ability of a material to allow light to pass through without significant scattering or absorption

      \item Some Ionic Compounds are Transparent

        \begin{itemize}

          \item Large Band Gaps: Ionic compounds like NaCl and MgO have large energy band gaps between their valence and conduction bands

          \item These gaps prevent absorption of visible light, allowing it to pass through
          
          \item Ordered Lattice: Regular arrangement of ions minimizes scattering, enhancing transparency

        \end{itemize}

    \end{itemize}

  \item Crystal Lattice Structures

    \begin{itemize}

      \item Formed to minimize energy by maximizing attractive forces and minimizing repulsive forces

      \item Crystal Lattice in Ionic Compounds:

        \begin{itemize}

          \item Cation-Anion Arrangement: Positive ions and negative ions alternate to create a stable structure

        \end{itemize}

      \item Types of Crystal Structures in Ionic Solids

        \begin{itemize}

          \item Simple Cubic: Cations and anions alternate in a simple cube arrangement

          \item Face-Centered Cubic (FCC): Each ion is surrounded by six oppositely charged ions in a cubic structure

        \end{itemize}

    \end{itemize}

  \item Lattice Energy and Stability

    \begin{itemize}

      \item What is Lattice Energy?

        \begin{itemize}

          \item The energy released when one mole of an ionic compound forms from its gaseous ions

          \item Represents the strength of bonds in the ionic lattice

          \item Higher lattice energy $\to$ stronger ionic bonds $\to$ greater stability

        \end{itemize}

      \item Factors Influencing Lattice Energy

        \begin{itemize}

          \item Ionic Charge: Higher charges increase electrostatic attraction

          \item Ionic Radius: Smaller ions result in shorter distances between ions, increasing attraction

        \end{itemize}

      \item Trends in Lattice Energy:

        \begin{itemize}

          \item Across a Period: Lattice energy increases as ionic radius decreases

          \item Down a Group: Lattice energy decreases as ionic radius increases

        \end{itemize}

    \end{itemize}

  \item Dielectric Materials in Capacitors

    \begin{itemize}

      \item What are Dielectric Materials?

        \begin{itemize}

          \item Insulating materials placed between the plates of a capacitor to increase its capacitance

          \item Dielectrics are often ionic or polar materials that can be polarized under an electric field

        \end{itemize}

      \item Role of Dielectrics in Capacitors

        \begin{itemize}

          \item Energy Storage

          \item Preventing Current Flow

        \end{itemize}

    \end{itemize}
    
  \item Insulating Materials

    \begin{itemize}

      \item What are Insulating Materials?

        \begin{itemize}

          \item Materials with high resistivity that prevent the flow of electric current

          \item Used to protect electrical components and ensure efficient energy transmission

        \end{itemize}

    \end{itemize}

\end{itemize}

\end{document}

