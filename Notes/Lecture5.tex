%%%%%%%%%%%%%%%%%%%%%%%%%%%%%%%%%%%%%%%%%%%%%%%%%%%%%%%%%%%%%%%%%%%%%%%%%%%%%%%%%%%%%%%%%%%%%%%%%%%%%%%%%%%%%%%%%%%%%%%%%%%%%%%%%%%%%%%%%%%%%%%%%%%%%%%%%%%%%%%%%%%
% Written By Michael Brodskiy
% Class: Electronic Materials
% Professor: J. Adams
%%%%%%%%%%%%%%%%%%%%%%%%%%%%%%%%%%%%%%%%%%%%%%%%%%%%%%%%%%%%%%%%%%%%%%%%%%%%%%%%%%%%%%%%%%%%%%%%%%%%%%%%%%%%%%%%%%%%%%%%%%%%%%%%%%%%%%%%%%%%%%%%%%%%%%%%%%%%%%%%%%%

\include{Includes.tex}

\title{Lecture 5 — Metallic Bonding}
\date{\today}
\author{Michael Brodskiy\\ \small Professor: J. Adams}

\begin{document}

\maketitle

\begin{itemize}

  \item Formation of Metallic Bonds

    \begin{itemize}

      \item Electron Delocalization: Metal atoms lose their valence electrons, which become part of a delocalized electron cloud

      \item Lattice Formation: Positive metal ions arrange themselves in a closely packed lattice structure

      \item Electrostatic Attraction: The ``sea of electrons'' binds the metal ions together, creating a cohesive and stable structure

    \end{itemize}

  \item Energy Bands in Metals

    \begin{itemize}

      \item In metals, atomic orbitals overlap to form continuous bands of energy levels

      \item Conduction Band: Contains energy levels where electrons are free to move and conduct electricity

      \item Valence Band: Occupied by electrons involved in bonding

    \end{itemize}

\end{itemize}

\end{document}

