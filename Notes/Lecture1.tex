%%%%%%%%%%%%%%%%%%%%%%%%%%%%%%%%%%%%%%%%%%%%%%%%%%%%%%%%%%%%%%%%%%%%%%%%%%%%%%%%%%%%%%%%%%%%%%%%%%%%%%%%%%%%%%%%%%%%%%%%%%%%%%%%%%%%%%%%%%%%%%%%%%%%%%%%%%%%%%%%%%%
% Written By Michael Brodskiy
% Class: Electronic Materials
% Professor: J. Adams
%%%%%%%%%%%%%%%%%%%%%%%%%%%%%%%%%%%%%%%%%%%%%%%%%%%%%%%%%%%%%%%%%%%%%%%%%%%%%%%%%%%%%%%%%%%%%%%%%%%%%%%%%%%%%%%%%%%%%%%%%%%%%%%%%%%%%%%%%%%%%%%%%%%%%%%%%%%%%%%%%%%

\documentclass[12pt]{article} 
\usepackage{alphalph}
\usepackage[utf8]{inputenc}
\usepackage[russian,english]{babel}
\usepackage{titling}
\usepackage{amsmath}
\usepackage{graphicx}
\usepackage{enumitem}
\usepackage{amssymb}
\usepackage[super]{nth}
\usepackage{expl3}
\usepackage[version=4]{mhchem}
\usepackage{hpstatement}
\usepackage{chemfig}
\usepackage{everysel}
\usepackage{ragged2e}
\usepackage{geometry}
\usepackage{multicol}
\usepackage{fancyhdr}
\usepackage{cancel}
\usepackage{siunitx}
\usepackage{physics}
\usepackage{tikz}
\usepackage{mathdots}
\usepackage{yhmath}
\usepackage{cancel}
\usepackage{color}
\usepackage{colortbl}
\usepackage{array}
\usepackage{multirow}
\usepackage{gensymb}
\usepackage{tabularx}
\usepackage{extarrows}
\usepackage{booktabs}
\usepackage{lastpage}
\usetikzlibrary{fadings}
\usetikzlibrary{patterns}
\usetikzlibrary{shadows.blur}
\usetikzlibrary{shapes}

\geometry{top=1.0in,bottom=1.0in,left=1.0in,right=1.0in}
\newcommand{\subtitle}[1]{%
  \posttitle{%
    \par\end{center}
    \begin{center}\large#1\end{center}
    \vskip0.5em}%

}
\usepackage{hyperref}
\hypersetup{
colorlinks=true,
linkcolor=blue,
filecolor=magenta,      
urlcolor=blue,
citecolor=blue,
}


\title{Lecture 1 — Bonding}
\date{\today}
\author{Michael Brodskiy\\ \small Professor: J. Adams}

\begin{document}

\maketitle

\begin{itemize}

  \item Covalent Bonds

    \begin{itemize}

      \item Bonds created by the structuring of valence electrons of atoms

    \end{itemize}

  \item Where are Covalent Bonds Found?

    \begin{itemize}

      \item Electronic semiconductor systems (Si, Ge, SiC, GaN, diamond)

      \item 2-D systems within the sheet (\textit{i}.\textit{e}. graphene)

      \item Wherever we find $3p^2$ and $3p^3$ bonds

    \end{itemize}

  \item Metallic Bonding

    \begin{itemize}

      \item Non-directional collective sharing of electrons

      \item Under an applied force, ions can move with respect to each other, especially when defects are present, hence metals are ductile

      \item Free valence electrons in the electron gas respond readily to applied electric fields and drift along the force of the field causing high electrical conductivity

    \end{itemize}

  \item Where are Metallic Bonds Found?

    \begin{itemize}

      \item Semiconductor system interconnect

      \item Metallization

      \item Bonding Technologies

    \end{itemize}

  \item Ionic Bonding

    \begin{itemize}

      \item Sodium Chloride (NaCl) is an example

        \begin{itemize}

          \item Sodium donates its single valence electron to complete chlorine's outer shell, causing it to be attracted via Coulombic forces

        \end{itemize}

      \item The electrostatic force of attraction between positive and negative ions that holds them together is called an ionic bond

      \item Ionic bonds are also called electrovalent bonds

      \item Oftentimes, a crystalline structure is formed

      \item Bonds are held at a potential energy minima

    \end{itemize}

  \item Determining Ionicity

    \begin{itemize}

      \item Electronegativity was introduced by Linus Pauling; Fluorine is assigned an electronegativity of 3.98, and other elements are scaled relative to that value

      \item Across a period (left to right), electronegativity increases due to increasing nuclear charge and decreasing atomic radius

      \item Down a group (top to bottom), electronegativity decreases due to increasing atomic radius and greater electron shielding

      \item Electronegativity difference determines bonds:

        \begin{itemize}

          \item Difference of 0-.39: Non-Polar Covalent Bond

          \item Difference of .4-1.69: Polar Covalent Bond

          \item Difference of 1.7 or more: Ionic Bond

        \end{itemize}

    \end{itemize}

  \item Where are Ionic Bonds Found?

    \begin{itemize}

      \item Solid-state ionic conductors are essential components of lithium-ion batteries

      \item Proton exchange membrane fuel cells (PEMFCs)

      \item Supercapacitors, a novel class of electrochemical energy storage devices

      \item Solid oxide fuel cells, devices that produce electricity from oxidizing fuel

    \end{itemize}

  \item Covalent versus Ionic Bonds: Covalent refers to a shared electron, while ionic refers to electron transfer

  \item Secondary Bonding and Van der Waals Bonds

    \begin{itemize}

      \item Covalent, metallic, and ionic are primary bonds

      \item Between all types of atoms and molecules there exists a weak attraction — the Van der Waals (London) Force

      \item In many molecules, the concentrations of $-$ and $+$ charges do not coincide

        \begin{itemize}

          \item This creates an electric dipole moment

        \end{itemize}

    \end{itemize}

  \item Where are Van der Waals Bonds Found?

    \begin{itemize}

      \item 2-D Electronic Systems (between layers)

    \end{itemize}

\end{itemize}

\end{document}

