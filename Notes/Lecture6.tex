%%%%%%%%%%%%%%%%%%%%%%%%%%%%%%%%%%%%%%%%%%%%%%%%%%%%%%%%%%%%%%%%%%%%%%%%%%%%%%%%%%%%%%%%%%%%%%%%%%%%%%%%%%%%%%%%%%%%%%%%%%%%%%%%%%%%%%%%%%%%%%%%%%%%%%%%%%%%%%%%%%%
% Written By Michael Brodskiy
% Class: Electronic Materials
% Professor: J. Adams
%%%%%%%%%%%%%%%%%%%%%%%%%%%%%%%%%%%%%%%%%%%%%%%%%%%%%%%%%%%%%%%%%%%%%%%%%%%%%%%%%%%%%%%%%%%%%%%%%%%%%%%%%%%%%%%%%%%%%%%%%%%%%%%%%%%%%%%%%%%%%%%%%%%%%%%%%%%%%%%%%%%

\include{Includes.tex}

\title{Lecture 6 — X-Rays}
\date{\today}
\author{Michael Brodskiy\\ \small Professor: J. Adams}

\begin{document}

\maketitle

\begin{itemize}

  \item Target Materials and Wavelengths

    \begin{itemize}

      \item Laboratory X-ray diffraction equipment relies on the use of an X-ray tube

    \end{itemize}

  \item Bragg's Law

    \begin{itemize}

      \item Refracted waves may be defined by:

        $$x+y=n\lambda$$
        $$d\sin(\theta)=x=y$$
        $$2d\sin(\theta)=n\lambda$$

        \begin{itemize}

          \item Where $n$ is the ``order'' of the wave ($n=1$ is the first harmonic), $x$ and $y$ are the distance between the angle of incidence and the original or refracted waves, and $\lambda$ is the wavelength

          \item Monochromators are used to limit the X-ray beam to be monochromatic in most cases

        \end{itemize}

    \end{itemize}

  \item Determining Inter-planar Spacing

    \begin{itemize}

      \item Cubic:

        $$d_{hkl}=\frac{a}{\sqrt{h^2+k^2+l^2}}$$

      \item Hexagonal:

        $$d_{hkl}=\frac{a}{\frac{4}{3}(h^2+k^2+hk)+\frac{a^2l^2}{c^2}}$$

    \end{itemize}

\end{itemize}

\end{document}

