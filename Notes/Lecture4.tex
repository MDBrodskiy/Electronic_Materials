%%%%%%%%%%%%%%%%%%%%%%%%%%%%%%%%%%%%%%%%%%%%%%%%%%%%%%%%%%%%%%%%%%%%%%%%%%%%%%%%%%%%%%%%%%%%%%%%%%%%%%%%%%%%%%%%%%%%%%%%%%%%%%%%%%%%%%%%%%%%%%%%%%%%%%%%%%%%%%%%%%%
% Written By Michael Brodskiy
% Class: Electronic Materials
% Professor: J. Adams
%%%%%%%%%%%%%%%%%%%%%%%%%%%%%%%%%%%%%%%%%%%%%%%%%%%%%%%%%%%%%%%%%%%%%%%%%%%%%%%%%%%%%%%%%%%%%%%%%%%%%%%%%%%%%%%%%%%%%%%%%%%%%%%%%%%%%%%%%%%%%%%%%%%%%%%%%%%%%%%%%%%

\documentclass[12pt]{article} 
\usepackage{alphalph}
\usepackage[utf8]{inputenc}
\usepackage[russian,english]{babel}
\usepackage{titling}
\usepackage{amsmath}
\usepackage{graphicx}
\usepackage{enumitem}
\usepackage{amssymb}
\usepackage[super]{nth}
\usepackage{expl3}
\usepackage[version=4]{mhchem}
\usepackage{hpstatement}
\usepackage{chemfig}
\usepackage{everysel}
\usepackage{ragged2e}
\usepackage{geometry}
\usepackage{multicol}
\usepackage{fancyhdr}
\usepackage{cancel}
\usepackage{siunitx}
\usepackage{physics}
\usepackage{tikz}
\usepackage{mathdots}
\usepackage{yhmath}
\usepackage{cancel}
\usepackage{color}
\usepackage{colortbl}
\usepackage{array}
\usepackage{multirow}
\usepackage{gensymb}
\usepackage{tabularx}
\usepackage{extarrows}
\usepackage{booktabs}
\usepackage{lastpage}
\usetikzlibrary{fadings}
\usetikzlibrary{patterns}
\usetikzlibrary{shadows.blur}
\usetikzlibrary{shapes}

\geometry{top=1.0in,bottom=1.0in,left=1.0in,right=1.0in}
\newcommand{\subtitle}[1]{%
  \posttitle{%
    \par\end{center}
    \begin{center}\large#1\end{center}
    \vskip0.5em}%

}
\usepackage{hyperref}
\hypersetup{
colorlinks=true,
linkcolor=blue,
filecolor=magenta,      
urlcolor=blue,
citecolor=blue,
}


\title{Lecture 4 — Covalent Bonding}
\date{\today}
\author{Michael Brodskiy\\ \small Professor: J. Adams}

\begin{document}

\maketitle

\begin{itemize}

  \item Electron Sharing in Covalent Bonds

    \begin{itemize}

      \item Covalent bonds form when atoms share electrons to achieve a stable electron configuration (octet rule)

      \item Key Characteristics

        \begin{itemize}

          \item Localized Electrons:

            \begin{itemize}

              \item Shared electrons are concentrated in the bond region between two nuclei

            \end{itemize}

          \item Directionality:

            \begin{itemize}

              \item Covalent bonds are directional, determining the geometry of molecules and latices

            \end{itemize}

        \end{itemize}

    \end{itemize}

  \item Octet Rule

    \begin{itemize}

      \item Atoms tend to share, gain, or lose electrons to achieve a stable configuration with eight valence electron configuration

      \item Covalent bonding enables atoms to fulfill this rule by sharing electrons

      \item Exceptions to the Octet Rule

        \begin{itemize}

          \item Expanded Octets: Elements like phosphorus (P) or sulfur (S) can have more than eight valence electrons

          \item Electron Deficiency: Atoms have fewer than 8 electrons in their valence shell

          \item Deviations occur due to varying atomic size, electron configurations, or bonding needs

          \item Odd-Electron Molecules (Radicals): Molecules with unpaired electrons, resulting in an incomplete octet

        \end{itemize}

    \end{itemize}

  \item Bond Types

    \begin{itemize}

      \item Single: Longest bond length and lowest bond energy

      \item Double: Shorter and stronger than single bonds

      \item Triple: Shortest bond length and highest bond energy

    \end{itemize}

  \item Covalent Networks

    \begin{itemize}

      \item Solids where atoms are connected in a continuous, 3D lattice through covalent bonds

      \item Distinguished from molecular solids, which consist of discrete molecules

    \end{itemize}
    
  \item Covalent Polymers

    \begin{itemize}

      \item Polymers are large molecules made up of repeating units (monomers) joined by covalent bonds

      \item Backbone of the polymer chain is typically a carbon-based structure, providing stability and flexibility

    \end{itemize}

  \item Molecular Orbital Theory

    \begin{itemize}

      \item A theory that explains covalent bonding by combining atomic orbitals to form molecular orbitals that are spread over the entire molecule

      \item Electrons are delocalized and occupy molecular orbitals, rather than being confined to individual bonds

      \item Principles:

        \begin{itemize}

          \item Constructive and Destructive Interference: Atomic Orbitals combine to form bonding (low energy) and anti-bonding (high energy) molecular orbitals

          \item Electron Filling: Molecular orbitals are filled from lower to highest energy levels, following the Pauli exclusion principle and Hind's rule

          \item Delocalization: Explains electron behavior in systems like graphene and benzene

        \end{itemize}

    \end{itemize}

  \item Hybridization

    \begin{itemize}

      \item A concept that explains how atomic orbitals mix to form new hybrid orbitals that are suitable for bonding

      \item Ensures maximum overlap and stronger bonds

    \end{itemize}

\end{itemize}

\end{document}

