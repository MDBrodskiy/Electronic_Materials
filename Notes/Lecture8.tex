%%%%%%%%%%%%%%%%%%%%%%%%%%%%%%%%%%%%%%%%%%%%%%%%%%%%%%%%%%%%%%%%%%%%%%%%%%%%%%%%%%%%%%%%%%%%%%%%%%%%%%%%%%%%%%%%%%%%%%%%%%%%%%%%%%%%%%%%%%%%%%%%%%%%%%%%%%%%%%%%%%%
% Written By Michael Brodskiy
% Class: Electronic Materials
% Professor: J. Adams
%%%%%%%%%%%%%%%%%%%%%%%%%%%%%%%%%%%%%%%%%%%%%%%%%%%%%%%%%%%%%%%%%%%%%%%%%%%%%%%%%%%%%%%%%%%%%%%%%%%%%%%%%%%%%%%%%%%%%%%%%%%%%%%%%%%%%%%%%%%%%%%%%%%%%%%%%%%%%%%%%%%

\include{Includes.tex}

\title{Lecture 8 — Semi-Conductors}
\date{\today}
\author{Michael Brodskiy\\ \small Professor: J. Adams}

\begin{document}

\maketitle

\begin{itemize}

  \item Intrinsic Semiconductors

    \begin{itemize}

      \item Energy Bands

      \item Electrons and Holes

      \item Conduction

      \item $e^-$ and $H^+$ Concentrations

    \end{itemize}

  \item Extrinsic Semiconductors

    \begin{itemize}

      \item Produced by doping

      \item Introduce small amounts of impurity atoms

      \item Semiconductor in which concentration of one type carrier is much greater than the other ($p>>n$ or $n>>p$)

      \item Pentavalent ions versus trivalent ions

    \end{itemize}

\end{itemize}

\end{document}

