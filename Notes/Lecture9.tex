%%%%%%%%%%%%%%%%%%%%%%%%%%%%%%%%%%%%%%%%%%%%%%%%%%%%%%%%%%%%%%%%%%%%%%%%%%%%%%%%%%%%%%%%%%%%%%%%%%%%%%%%%%%%%%%%%%%%%%%%%%%%%%%%%%%%%%%%%%%%%%%%%%%%%%%%%%%%%%%%%%%
% Written By Michael Brodskiy
% Class: Electronic Materials
% Professor: J. Adams
%%%%%%%%%%%%%%%%%%%%%%%%%%%%%%%%%%%%%%%%%%%%%%%%%%%%%%%%%%%%%%%%%%%%%%%%%%%%%%%%%%%%%%%%%%%%%%%%%%%%%%%%%%%%%%%%%%%%%%%%%%%%%%%%%%%%%%%%%%%%%%%%%%%%%%%%%%%%%%%%%%%

\documentclass[12pt]{article} 
\usepackage{alphalph}
\usepackage[utf8]{inputenc}
\usepackage[russian,english]{babel}
\usepackage{titling}
\usepackage{amsmath}
\usepackage{graphicx}
\usepackage{enumitem}
\usepackage{amssymb}
\usepackage[super]{nth}
\usepackage{expl3}
\usepackage[version=4]{mhchem}
\usepackage{hpstatement}
\usepackage{chemfig}
\usepackage{everysel}
\usepackage{ragged2e}
\usepackage{geometry}
\usepackage{multicol}
\usepackage{fancyhdr}
\usepackage{cancel}
\usepackage{siunitx}
\usepackage{physics}
\usepackage{tikz}
\usepackage{mathdots}
\usepackage{yhmath}
\usepackage{cancel}
\usepackage{color}
\usepackage{colortbl}
\usepackage{array}
\usepackage{multirow}
\usepackage{gensymb}
\usepackage{tabularx}
\usepackage{extarrows}
\usepackage{booktabs}
\usepackage{lastpage}
\usetikzlibrary{fadings}
\usetikzlibrary{patterns}
\usetikzlibrary{shadows.blur}
\usetikzlibrary{shapes}

\geometry{top=1.0in,bottom=1.0in,left=1.0in,right=1.0in}
\newcommand{\subtitle}[1]{%
  \posttitle{%
    \par\end{center}
    \begin{center}\large#1\end{center}
    \vskip0.5em}%

}
\usepackage{hyperref}
\hypersetup{
colorlinks=true,
linkcolor=blue,
filecolor=magenta,      
urlcolor=blue,
citecolor=blue,
}


\title{Lecture 9 — Dielectrics}
\date{\today}
\author{Michael Brodskiy\\ \small Professor: J. Adams}

\begin{document}

\maketitle

\begin{itemize}

  \item Capacitance

    $$C=\frac{\varepsilon_r\varepsilon_oA}{d}$$

    \begin{itemize}

      \item $\varepsilon_o$ represents the permittivity of free space

      \item $\varepsilon_r$ is the permittivity of a dielectric material

      \item $A$ is the cross-sectional area of the plates

      \item $d$ is the distance between plates

      \item $C$ represents the effective capacitance

      \item Add in parallel, divide in series

    \end{itemize}

  \item Dielectric Strength

    \begin{itemize}

      \item When the electric field in the dielectric reaches a critical value called the dielectric strength, the medium suffers a dielectric breakdown where a large current flows across the plates

    \end{itemize}

  \item Dielectric Theory

    \begin{itemize}

      \item $p$ is the electric dipole moment (measure of electrostatic effect of opposite charges displaced by a): $p=Qa$

      \item Power dissipation in a capacitor occurs and is also frequency dependent

    \end{itemize}

  \item Polarizability

    \begin{itemize}

      \item Polarizability is defined as:

        $$p=\alpha E$$

      \item The induced dipole moment is called the electronic polarizability, $\alpha_e$

      \item The electronic polarization may be written as:

        $$p_e=\left( \frac{Z^2e^2}{\beta} \right)$$

      \item We can find susceptibility as:

        $$\chi_e=\frac{1}{\varepsilon_o}N\alpha_e$$

      \item The two are related through:

        $$\varepsilon_r=1+\chi_e$$

    \end{itemize}

  \item Clausius-Mossotti Equation

    \begin{itemize}

      \item Taking:

        $$p=\alpha_eE$$
        $$P=\chi_e\varepsilon_oE$$
        $$\varepsilon_r=1+\chi_e$$

      \item We combine all of the relationships to get:

        $$\frac{\varepsilon_r-1}{\varepsilon_r+2}=\frac{N\alpha_e}{3\varepsilon_o}$$

    \end{itemize}

  \item Relaxation

    \begin{itemize}

      \item The rate of polarization change may be written as:

        $$\frac{dp}{dt}=-\frac{p}{\tau}+\frac{\alpha_d(o)}{\tau}E_oe^{j\omega t}$$

      \item We can solve to obtain:

        $$p=\alpha_d(\omega)E_oe^{j\omega t}$$

    \end{itemize}

\end{itemize}

\end{document}

