\documentclass[conference]{IEEEtran}
\IEEEoverridecommandlockouts
% The preceding line is only needed to identify funding in the first footnote. If that is unneeded, please comment it out.
%Template version as of 6/27/2024

\usepackage{cite}
\usepackage{siunitx}
\usepackage{hyperref}
\hypersetup{
colorlinks=true,
linkcolor=blue,
filecolor=magenta,      
urlcolor=blue,
citecolor=blue,
}
\usepackage{amsmath,amssymb,amsfonts}
\usepackage{algorithmic}
\usepackage{graphicx}
\usepackage{textcomp}
\usepackage{xcolor}
\def\BibTeX{{\rm B\kern-.05em{\sc i\kern-.025em b}\kern-.08em
    T\kern-.1667em\lower.7ex\hbox{E}\kern-.125emX}}
\begin{document}

\title{Two-Dimensional Materials in Electronics\\
}

\author{\IEEEauthorblockN{Michael Brodskiy}
\IEEEauthorblockA{\textit{College of Engineering} \\
\textit{Northeastern University}\\
Boston, MA\\
\href{mailto:Brodskiy.M@Northeastern.edu}{Brodskiy.M@Northeastern.edu}}
\and
\IEEEauthorblockN{Thomas Czartoryski}
\IEEEauthorblockA{\textit{Khoury College of Computer Science} \\
\textit{Northeastern University}\\
Boston, MA\\
\href{mailto:Czartoryski.T@Northeastern.edu}{Czartoryski.T@Northeastern.edu}}
\and
\IEEEauthorblockN{Oluwalaanu Adeboye}
\IEEEauthorblockA{\textit{College of Engineering} \\
\textit{Northeastern University}\\
Boston, MA\\
\href{mailto:Adeboye.O@Northeastern.edu}{Adeboye.O@Northeastern.edu}}
\and
\IEEEauthorblockN{Daniela Salazar}
\IEEEauthorblockA{\textit{College of Engineering} \\
\textit{Northeastern University}\\
Boston, MA\\
\href{mailto:Salazar.Dani@Northeastern.edu}{Salazar.Dani@Northeastern.edu}}
\and
\IEEEauthorblockN{John Bergin}
\IEEEauthorblockA{\textit{College of Engineering} \\
\textit{Northeastern University}\\
Boston, MA\\
\href{mailto:Bergin.J@Northeastern.edu}{Bergin.J@Northeastern.edu}}
\and
\IEEEauthorblockN{Owen Chiu}
\IEEEauthorblockA{\textit{College of Engineering} \\
\textit{Northeastern University}\\
Boston, MA\\
\href{mailto:Chiu.O@Northeastern.edu}{Chiu.O@Northeastern.edu}}
}

\maketitle

\begin{abstract}
  The following is a meta-analysis which delves into the field of two-dimensional (2D) materials within the realm of electronics. Key materials under scrutiny include graphene, molybdenum disulfide (MoS$_2$), and hexagonal boron nitride (hBN). This document and the research herein will then be used to generate an accompanying presentation.
\end{abstract}

\begin{IEEEkeywords}
  \underline{meta-analysis}, \underline{two-dimensional materials}, \underline{electronics}
\end{IEEEkeywords}

\section{Introduction}

\section{Graphene}

\subsection{Material Structure}

The atomic structure of graphene consists of a single layer of carbon atoms in the formation of a hexagonal lattice. This structure makes graphene one of the most promising materials within the realm of electronics, as it allows for exceptional electrical, mechanical, and thermal properties \cite{b1}. These properties contribute to graphene's high carrier mobility and physical strength, which has led to its integration into transistors, and, subsequently, touch screen and energy storage devices \cite{b2}.

\subsection{Electromagnetic Properties}

The structure of graphene permits remarkable electrical conductivity (several orders of magnitude higher than silicon), with a carrier mobility of $200,000\left[ \si{\cm\squared\per\volt\second} \right]$ at room temperature \cite{b3}. The electrons within the graphene exhibit Dirac fermion behavior, which is the primary reason for such mobility \cite{b4}.

\subsection{Mechanical Properties}

Due to the hexagonal lattice, graphene is ideal for electrical applications which require robust physical properties. The tensile strength of the material is approximately $130[\si{\giga\pascal}]$ \cite{b5}; furthermore, graphene's flexibility makes it suitable for cases in which the device may experience stretching or warping \cite{b6}.

\subsection{Thermal Properties}

In addition to its electrical conductivity and physical strength, graphene holds a high thermal conductivity. At room temperature, graphene commands a thermal conductivity of approximately $5,300[\si{\watt\per\milli\kelvin}]$. As such, it may be used for thermal management \cite{b7}.

\subsection{The Future of Graphene and Electrical Devices}

Though the material itself is highly suitable for use within electronic devices, there are several challenges beyond aptitude which make the incorporation of graphene difficult. First and foremost is the difficulty of large-scale production. This, coupled with the difficulty of controlling the electrical properties and development of effect contact materials, make graphenes applicability within electronic materials uncertain \cite{b8}.

\section{Hexagonal Boron Nitride (hBN)}

\section*{References}

The following works were referenced within this meta-analysis.

\begin{thebibliography}{00}
\bibitem{b1} Novoselov, K. S., Geim, A. K., Morozov, S. V., Jiang, D., Zhang, Y., Dubonos, S. V., ... & Firsov, A. A. (2004). Electric field effect in atomically thin carbon films. Science, 306(5696), 666-669.
\bibitem{b2} Geim, A. K., & Novoselov, K. S. (2007). The rise of graphene. Nature materials, 6(3), 183-191.
\bibitem{b3} Bolotin, K. I., Sikes, K. J., Jiang, Z., Fudenberg, G., Hone, J., Kim, P., & Stormer, H. L. (2008). Ultrahigh electron mobility in suspended graphene. Solid State Communications, 146(1-3), 351-355.
\bibitem{b4} Castro Neto, A. H., Guinea, F., Peres, N. M. R., Novoselov, K. S., & Geim, A. K. (2009). The electronic properties of graphene. Reviews of modern physics, 81(1), 109.
\bibitem{b5} Balandin, A. A., Ghosh, S., Bao, W., Calizo, I., Teweldebrhan, D., Miao, F., & Lau, C. N. (2008). Superior thermal conductivity of single-layer graphene. Nano letters, 8(3), 902-907.
\bibitem{b6} Wang, X., Li, Q., & Zhang, Y. (2012). Graphene-based flexible electronics. Journal of materials chemistry, 22(13), 5965-5978.
\bibitem{b7} Balandin, A. A. (2008). Thermal conductivity of graphene and graphene-based composites. Nano letters, 8(11), 3435-3437.
\bibitem{b8} Li, X., Wang, X., Zhang, Y., & Bao, W. (2010). Graphene: A versatile material for flexible electronics. Advanced Materials, 22(35), 3877-3890.
\end{thebibliography}

\end{document}
